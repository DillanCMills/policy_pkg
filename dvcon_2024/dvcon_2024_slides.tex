\documentclass[aspectratio=169]{beamer}
\usepackage{tikz}
\usepackage{graphicx}
\usepackage{helvet}
\usepackage{minted}
\renewcommand{\MintedPygmentize}{./pygments-plugin-systemverilog/venv/bin/pygmentize}
\usepackage[framemethod=TikZ]{mdframed}
\usepackage{hyperref}
\renewcommand{\thefootnote}{\fnsymbol{footnote}}

\mdfsetup{
    middlelinewidth   = 2,
    roundcorner       = 10pt,
    leftmargin        = -10pt,
    rightmargin       = -10pt,
    backgroundcolor   = gray!05,
    middlelinecolor   = gray!30!white,
    innertopmargin    = 0.1cm,
    innerbottommargin = 0.1cm,
    splittopskip      = \topskip,
}

\usemintedstyle{sv-style-light}

\newcommand{\svfile}[2]{{
\begin{mdframed}
\inputminted[
    linenos,
    numbersep=16pt,
    breaklines,
]{sv-lang}{#1}
\end{mdframed}
\ifthenelse{\equal{#2}{}}{}{\captionof{figure}{#2}}
}}

\newcommand{\svfilerange}[4]{{
\begin{mdframed}
\inputminted[
    linenos,
    numbersep=16pt,
    breaklines,
    firstline=#2,
    lastline=#3,
    firstnumber=1,
]{sv-lang}{#1}
\end{mdframed}
\ifthenelse{\equal{#4}{}}{}{\captionof{figure}{#4}}
}}

\newcommand{\code}[1]{
  \textbf{\mintinline{text}{#1}}
}

\newminted[svcode]{sv-lang}{
    linenos,
    numbersep=16pt,
    breaklines,
}

\newminted[pycode]{python}{
    linenos,
    numbersep=16pt,
    breaklines,
}

\surroundwithmdframed{svcode}
\surroundwithmdframed{pycode}

\usetheme{default}
\setbeamertemplate{itemize items}{\color{black}$\bullet$}
\setbeamertemplate{footline}{\vspace{1.4cm}}
\setbeamertemplate{navigation symbols}{}

\setbeamercolor{frametitle}{fg=black}
\setbeamerfont{frametitle}{size=\huge}
\setbeamertemplate{frametitle}{\vspace{0.5cm}\insertframetitle}

% Footer
\setbeamertemplate{background}{
  \begin{tikzpicture}[overlay, remember picture]
    \node[] at (current page.center){
      \includegraphics[height=\paperheight]{images/image1.png}
    };
    \node[above right=0.1cm] (logo2) at (current page.south west) {
      \includegraphics[height=0.8cm]{images/image2.png}
    };
    \node[below left=0.1cm] (page number) at (current page.north east) {
      \tiny{\insertframenumber}
    };
  \end{tikzpicture}
}

\begin{document}

% Title page
{
  \setbeamertemplate{background}{
    \begin{tikzpicture}[overlay, remember picture]
      \node[] at (current page.center){
        \includegraphics[height=\paperheight]{images/image3.png}
      };
    \end{tikzpicture}
  }
  \begin{frame}
    \begin{tikzpicture}[overlay, remember picture]
    \linespread{2.2}
    \node[below=0.5cm] (logo) at (current page.north) {
      \includegraphics[height=4cm]{images/image4.png}
    };
    \node[below, align=center, minimum width=0.7\textwidth, text width=1\textwidth] (title) at (logo.south) {
      {\color{white}\Huge{Four Problems with Policy-Based Constraints and How to Fix Them}}
    };
    \linespread{1}
    \node[below left, inner xsep=15pt, inner ysep=10pt, text width=0.3\textwidth] (author1) at (title.south){
      {\color{lightgray}\large {Dillan Mills} \\[5pt]
      \large{Synopsys, Inc.}}
    };
    \node[below right, inner xsep=15pt, inner ysep=10pt, text width=0.3\textwidth] (author2) at (title.south){
      {\color{lightgray}\large {Chip Haldane} \\[5pt]
      \large{The Chip Abides, LLC}}
    };
    \node[above left=0.4cm] (logo2) at (current page.south east) {
      \includegraphics[height=0.8cm]{images/image2.png}
    };
    \end{tikzpicture}
  \end{frame}
}

% Constraints and Policy Class Review
\begin{frame}
  \begin{tikzpicture}[overlay, remember picture]
    \node[] at (current page.center){
      \includegraphics[height=\paperheight]{images/image5.png}
    };
    \node[align=center, minimum width=0.7\textwidth, text width=1\textwidth] (title) at (current page.center) {
      {\color{white}\Huge{Constraints and Policy Class Review}}
    };
  \end{tikzpicture}
\end{frame}

\begin{frame}{Constraints Review}
  \begin{itemize}
    \setlength\itemsep{1em}
    \item Random objects and constraints are the foundational building blocks of constrained random verification
    \item Embedded fixed constraints are simple but lack flexibility
    \item In-line constraints are marginally more flexible but their definitions are still fixed within the calling context
    \item In-line constraints must all be specified within a single call to \code{randomize()} 
  \end{itemize}
\end{frame}

\begin{frame}{Policy Class Review}
  \begin{itemize}
    \setlength\itemsep{1em}
    \item Policy classes are a technique for applying constraints in a portable, reusable, and incremental manner
    \item Leverage an aspect of "global constraints", simultaneously solving constraints across a set of random objects
    \item Randomizing a class that contains policies also randomizes the policies
    \item The policies contain a reference back to the container
    \item Consequently, the policy container is constrained by the policies it contains
  \end{itemize}
\end{frame}

\begin{frame}{Policy Class Example: \code{policy_base}}
\svfilerange{example_code/v1_original_policies/policy_base.svh}{1}{7}{}
\end{frame}

\begin{frame}{Policy Class Example: \code{policy_list}}
\svfilerange{example_code/v1_original_policies/policy_base.svh}{9}{19}{}
\end{frame}

\begin{frame}{\code{policy_base} and \code{policy_list} Summary}
\begin{columns}
\column{1.01\textwidth}
  \begin{itemize}
    \setlength\itemsep{1em}
    \item These two base classes provide the core definitions for policies
    \item \code{policy_base} implements the hook back to the policy container
    \item \code{policy_list} organizes related policies into groups
    \item Parameterization requires a unique specialization per policy-enabled container
  \end{itemize}
\end{columns}
\end{frame}

\begin{frame}[fragile]{Policy Class Example: Implementation}
\scriptsize
\begin{columns}
\begin{column}{0.45\textwidth}
\begin{svcode}
class addr_txn;
  rand addr_t addr;
  rand policy_base#(addr_txn) policy[$];

  function void pre_randomize;
    foreach(policy[i])
      policy[i].set_item(this);
  endfunction
endclass
\end{svcode}
\end{column}
\begin{column}{0.45\textwidth}
\begin{svcode}
class addr_policy extends policy_base#(addr_txn);
  rand addr_t addrs[$];

  function void add(addr_t addr);
    addrs.push_back(addr);
  endfunction

  constraint c_addr {
    item.addr inside {addrs};
  }
endclass
\end{svcode}
\vspace{0.2cm}
\end{column}
\end{columns}
\normalsize
\begin{itemize}
\item \code{addr_txn.addr} is constrained to a value added through \code{addr_policy}
\end{itemize}
\end{frame}

\begin{frame}[fragile]{Policy Class Example: Usage}
\begin{columns}
\begin{column}{0.65\textwidth}
\begin{svcode}
class addr_constrained_txn extends addr_txn;
  function new;
    addr_policy addr_policy = new;
    policy_list#(addr_txn) pcy = new;
    addr_policy.add('h00000000);
    addr_policy.add('h10000000);
    pcy.add(addr_policy);
    this.policy = {pcy};
  endfunction
endclass
\end{svcode}
\end{column}
\begin{column}{0.3\textwidth}
  \begin{itemize}
    \item Instantiate and randomize like normal with a call to \code{txn.randomize()}
    \item Each value added to the policy list will constrain \code{addr}
  \end{itemize}
\end{column}
\end{columns}
\end{frame}

% Problem 1: Parameterized Policies
\begin{frame}
  \begin{tikzpicture}[overlay, remember picture]
    \node[] at (current page.center){
      \includegraphics[height=\paperheight]{images/image5.png}
    };
    \node[align=center, minimum width=0.7\textwidth, text width=1\textwidth] (title) at (current page.center) {
      {\color{white}\Huge{Problem \#1: Parameterized Policies}}
    };
  \end{tikzpicture}
\end{frame}

\begin{frame}{Problem \#1: Parameterized Policies}
  \begin{itemize}
    \setlength\itemsep{1em}
    \item \code{policy_base} is parameterized to the class it constrains
    \item Different specializations cannot be grouped and indexed
    \item An extended class hierarchy with constrainable values in each level requires a unique policy type and policy list for each level
    \item Each list must be separately traversed and mapped back to the container during \code{pre_randomize}
    \item Users have to keep track of the different lists and which signals belong to each
  \end{itemize}
\end{frame}

\begin{frame}[fragile]{Parameterized Policies: Example}
\scriptsize
\begin{columns}
\begin{column}{0.45\textwidth}
\begin{svcode}
class addr_p_txn extends addr_txn;
  rand bit parity;
  rand policy_base#(addr_p_txn) p_policy[$];

  function void pre_randomize;
    foreach(p_policy[i])
      p_policy[i].set_item(this);
  endfunction
endclass
\end{svcode}
\end{column}
\begin{column}{0.45\textwidth}
\begin{svcode}
class addr_c_txn extends addr_p_txn;
  function new;
    policy_list#(addr_txn) pcy = new;
    policy_list#(addr_p_txn) p_pcy = new;
    pcy.add(/*addr policies*/);
    p_pcy.add(/*parity policies*/);
    this.policy = {pcy};
    this.p_policy = {p_pcy};
  endfunction
endclass
\end{svcode}
\end{column}
\end{columns}
\vspace{0.2cm}
\normalsize
\begin{itemize}
\item This method will not scale---each additional subclass requires a new policy type and list
\end{itemize}
\end{frame}

\begin{frame}[fragile]{Parameterized Policies: Solution}
\begin{columns}
\begin{column}{0.3\textwidth}
\begin{itemize}
  \setlength\itemsep{1em}
  \item Replace the parameterized policy base with a non-parameterized base
  \item Move parameters to an extension class that implements the interface
\end{itemize}
\end{column}
\begin{column}{0.6\textwidth}
\scriptsize
\begin{svcode}
interface class policy;
  pure virtual function void set_item(uvm_object item);
endclass

virtual class policy_imp#(type ITEM=uvm_object) implements policy;
  protected rand ITEM m_item;

  virtual function void set_item(uvm_object item);
    if (!$cast(m_item, item)) /* cleanup */;
  endfunction
endclass

typedef policy policy_queue[$];
\end{svcode}
\end{column}
\end{columns}
\end{frame}

\begin{frame}{Policy Interface and Implementation Classes}
\begin{itemize}
  \setlength\itemsep{1em}
  \item Non-parameterized base enables all policies targeting a particular class hierarchy to be stored in a single common \code{policy_queue}
  \item Parameterized \code{policy_imp} implements the base and provides core functionality required by all policies
  \item No strong typing means all implementing classes must share a common base class---\code{uvm_object} is a safe choice for UVM testbenches
  \item \code{set_item()} must handle the cases where an invalid type is passed in
\end{itemize}
\end{frame}

\begin{frame}[fragile]{Policy Definition Updates}
\begin{itemize}
  \setlength\itemsep{1em}
  \item Policy definitions are mostly still the same as original classes
  \item Policy classes should be updated to extend the new \code{policy_imp} class
\end{itemize}
\begin{svcode}
class addr_policy extends policy_imp#(addr_txn);
\end{svcode}
\begin{itemize}
  \item Constraints should be written as implications in case \code{item} is missing
  \end{itemize}
\begin{svcode}
constraint c_addr {m_item != null -> m_item.addr inside {addrs};}
\end{svcode}
\end{frame}

\begin{frame}{Policy Implementation and Usage Updates}
\begin{itemize}
  \setlength\itemsep{1em}
  \item The base txn class needs to inherit from \code{uvm_object} to be type-compatible
  \item The \code{policies} list in the base txn is replaced with a \code{rand policy_queue policies} declaration
  \item Subclasses of the base txn class no longer need their own policy lists or \code{pre_randomize()} functions
  \item The constrained txn can push all policies into the shared \code{policy_queue} in the base txn class
\end{itemize}
\end{frame}

\begin{frame}[fragile]{Policy Usage Example}
\begin{svcode}
class addr_txn extends uvm_object;
  rand policy_queue policies;
// ...
class addr_c_txn extends addr_p_txn;
  function new;
    // ...
    this.policies.push_back(/*addr_txn policies*/);
    this.policies.push_back(/*addr_p_txn policies*/);
  endfunction
endclass
\end{svcode}
\end{frame}

% Problem 2: Definition Location
\begin{frame}
  \begin{tikzpicture}[overlay, remember picture]
    \node[] at (current page.center){
      \includegraphics[height=\paperheight]{images/image5.png}
    };
    \node[align=center, minimum width=0.7\textwidth, text width=1\textwidth] (title) at (current page.center) {
      {\color{white}\Huge{Problem \#2: Definition Location}}
    };
  \end{tikzpicture}
\end{frame}

\begin{frame}{Problem \#2: Definition Location}
  \begin{itemize}
    \setlength\itemsep{1em}
    \item ``Where should I define my policy classes?''
    \item Easy enough to stick them in a file close to the class they are constraining
    \item Better solution: directly embed policy definitions in the class they constrain
    \vspace{8pt}
    \begin{itemize}
      \setlength\itemsep{1em}
      \item Eliminates all guesswork about where to define and discover policies
      \item Embedded policies gain access to all members of their container class (including protected properties and methods)
    \end{itemize}
  \end{itemize}
\end{frame}

\begin{frame}[fragile]{Embedded Policy Example}
\scriptsize
\begin{columns}
\begin{column}{0.45\textwidth}
\begin{svcode}
class addr_txn extends uvm_object;
  class POLICIES;
    /* policy definitions */
  endclass
endclass

class addr_p_txn extends addr_txn;
  class POLICIES extends addr_txn::POLICIES;
    /* additional policy definitions */
  endclass
endclass
\end{svcode}
\end{column}
\begin{column}{0.45\textwidth}
\begin{svcode}
class addr_c_txn extends addr_p_txn;
  function new;
    addr_c_txn::POLICIES::addr_policy a_pcy = new(/*...*/);
    this.policies.push_back(a_pcy);
  endfunction
endclass
\end{svcode}
\normalsize
\begin{itemize}
  \item Wrap the policies in a \code{POLICIES} class to optimize organization
  \item Subclass \code{POLICIES} extend from parent \code{POLICIES}
\end{itemize}
\end{column}
\end{columns}
\end{frame}

\begin{frame}[fragile]{Optimize Further}
\begin{columns}
\begin{column}{0.40\textwidth}
\begin{itemize}
  \setlength\itemsep{1em}
  \item Mark properties \code{protected} so they can only be manipulated with policies
  \item Add static functions to instantiate and initialize policies with a single call
\end{itemize}
\end{column}
\begin{column}{0.55\textwidth}
\scriptsize
\begin{svcode}
class addr_txn extends uvm_object;
  protected rand a_t addr;
  class POLICIES;
    // ... addr_policy definition
    static function addr_policy FIXED_ADDR(a_t a);
      FIXED_ADDR = new(a);
    endfunction
  endclass
endclass

class addr_c_txn extends addr_p_txn;
  function new;
    this.policies.push_back(
      addr_c_txn::POLICIES::FIXED_ADDR('hFF00));
  endfunction
endclass
\end{svcode}
\end{column}
\end{columns}
\end{frame}

% Problem 3: Boilerplate Overload
\begin{frame}
  \begin{tikzpicture}[overlay, remember picture]
    \node[] at (current page.center){
      \includegraphics[height=\paperheight]{images/image5.png}
    };
    \node[align=center, minimum width=0.7\textwidth, text width=1\textwidth] (title) at (current page.center) {
      {\color{white}\Huge{Problem \#3: Boilerplate Overload}}
    };
  \end{tikzpicture}
\end{frame}

\begin{frame}[fragile]{Problem \#3: Boilerplate Overload}
  \begin{itemize}
    \item Policies require a class definition, a constructor, and a constraint (at a minimum)
    \item Relatively unavoidable for complex policies
    \item Generic policy types that show up a lot can be simplified with a macro
    \item Use macros to set up the embedded \code{POLICIES} class, the policy definition, and a static constructor
  \end{itemize}
  \scriptsize
\begin{columns}
\begin{column}{0.6\textwidth}
\begin{svcode}
class addr_p_txn extends addr_txn;
  `start_extended_policies(addr_p_txn, addr_txn)
    `fixed_policy(PARITY_ERR, parity_err, bit)
  `end_policies
endclass
\end{svcode}
\end{column}
\end{columns}
\end{frame}

\begin{frame}{Policy Macros}
\begin{itemize}
  \setlength\itemsep{1em}
  \item Ideal for properties with simple constraints, such as equality, range, or set membership constraints
  \item Complex constraints with relationships between multiple properties are harder to turn into a macro
  \vspace{8pt}
  \begin{itemize}
    \setlength\itemsep{1em}
    \item They can be left as-is, defined within the \code{POLICIES} class
    \item Define in a separate file and add to the \code{POLICIES} class with \code{`include}
  \end{itemize}
\end{itemize}
\end{frame}

% Problem 4: Unexpected Policy Reuse Behavior and Optimizing for Lightweight Policies
\begin{frame}
  \begin{tikzpicture}[overlay, remember picture]
    \linespread{2.5}
    \node[] at (current page.center){
      \includegraphics[height=\paperheight]{images/image5.png}
    };
    \node[align=center, minimum width=0.7\textwidth, text width=1\textwidth] (title) at (current page.center) {
      {\color{white}\Huge{Problem \#4: Unexpected Policy Reuse Behavior and Optimizing for Lightweight Policies}}
    };
  \end{tikzpicture}
\end{frame}

\begin{frame}{\LARGE{Problem \#4: Unexpected Policy Reuse Behavior and Optimizing for Lightweight Policies}}
  \begin{itemize}
    \setlength\itemsep{1em}
    \item Occasional unexpected behavior when re-randomizing objects with policies
    \item Policies ``remember'' previous randomizations and wouldn't reapply constraints
    \item Sidestep the issue---keep policy classes extremely lightweight and adopt a ``use once and discard'' approach
    \item Introduce a \code{copy()} method that returns a fresh policy instance initialized to the same state as the policy that implements it
    \item Rely on static constructors to generate initialized policies automatically
  \end{itemize}
\end{frame}

% More Improvements to the Policy Package
\begin{frame}
  \begin{tikzpicture}[overlay, remember picture]
    \linespread{2.5}
    \node[] at (current page.center){
      \includegraphics[height=\paperheight]{images/image5.png}
    };
    \node[align=center, minimum width=0.7\textwidth, text width=1\textwidth] (title) at (current page.center) {
      {\color{white}\Huge{More Improvements to the Policy Package}}
    };
  \end{tikzpicture}
\end{frame}

\begin{frame}{More Improvements to the Policy Package}
  \begin{itemize}
    \setlength\itemsep{1em}
    \item Lots of nice improvements to the original policy package so far
    \item Still lacking many features that would be useful in real-world implementations
    \item Refer to the paper for complete code examples and a more detailed discussion of the following features
  \end{itemize}
\end{frame}

\begin{frame}{Expanding the \code{policy} interface class}
\svfile{../policy_pkg/policy.svh}{}
\end{frame}

\begin{frame}[fragile]{Better type safety checking and reporting in \code{policy_imp} methods}
\scriptsize
\begin{svcode}
virtual function void set_item(uvm_object item);
    if (item == null) 
        `uvm_error(/* NULL item passed */)
    else if ((this.item_is_compatible(item)) && $cast(this.m_item, item)) begin
        `uvm_info(/* Policy applied */)
        this.m_item.rand_mode(1);
    end else begin
        `uvm_warning(/* Incompatible type */)
        this.m_item = null;
        this.m_item.rand_mode(0);
    end
endfunction
\end{svcode}
\end{frame}

\begin{frame}[fragile]{\LARGE{Replacing \code{policy_list} with \code{policy_queue}}}
\begin{itemize}
  \setlength\itemsep{0.8em}
  \item The original policy implementation used a \code{policy_list} to hold policies
  \item With the new implementation, all you need is a typedef queue
\end{itemize}
\begin{columns}
\begin{column}{0.5\textwidth}
\vspace{-0.2cm}
\begin{svcode}
typedef policy policy_queue[$];
\end{svcode}
\vspace{0.1cm}
\end{column}
\end{columns}
\begin{itemize}
  \setlength\itemsep{0.8em}
  \item A \code{policy_queue} can hold any policy that implements the \code{policy} interface
  \item Default queue methods can be used to aggregate policies
  \item Array literals can be used where a \code{policy_queue} is expected
  \item Define, initialize, aggregate, and pass policies all in a single line of code!
\end{itemize}
\end{frame}

\begin{frame}[fragile]{\Large{Standardize policy implementations with the \code{policy_container} interface and \code{policy_object} mixin}}
\begin{svcode}
interface class policy_container;
  pure virtual function bit has_policies();

  pure virtual function void set_policies(policy_queue policies);
  pure virtual function void add_policies(policy_queue policies);
  pure virtual function void clear_policies();

  pure virtual function policy_queue get_policies();

  pure virtual function policy_queue copy_policies();
endclass
\end{svcode}
\end{frame}

\begin{frame}[fragile]{Using the \code{policy_container} interface class and \code{policy_object} mixin}
\scriptsize
\begin{svcode}
class policy_object #(type BASE=uvm_object) extends BASE implements policy_container;
  protected policy_queue m_policies;
  // ... fill out policy_container functions
endclass
\end{svcode}
\begin{svcode}
// Use policy_object for transactions
class base_txn extends policy_object #(uvm_sequence_item);

// Use policy_object for sequences
class base_seq #(type REQ=uvm_sequence_item, RSP=REQ) extends policy_object #( uvm_sequence#(REQ, RSP) );

// Use policy_object for configuration objects
class cfg_object extends policy_object #(uvm_object);
\end{svcode}
\end{frame}

\begin{frame}[fragile]{Protecting the policy queue enforces loosely coupled code}
\begin{itemize}
  \item Using \code{policy_object} along with \code{policy_container} allows the policy queue to be protected to prevent direct access
  \item Original implementation required calling \code{set_item} on each policy during \code{pre_randomize}
  \begin{itemize}
    \item Required because \code{policy_list} was public so callers could add policies without linking to the target class
  \end{itemize}
  \item Making it private forces callers to set or add policies with the exposed interface methods
  \item These methods can also check compatibility and call \code{set_item} immediately
  \item Removes the reliance on \code{pre_randomize}
\end{itemize}
\end{frame}

% Conclusion
\begin{frame}
  \begin{tikzpicture}[overlay, remember picture]
    \node[] at (current page.center){
      \includegraphics[height=\paperheight]{images/image5.png}
    };
    \node[align=center, minimum width=0.7\textwidth, text width=1\textwidth] (title) at (current page.center) {
      {\color{white}\Huge{Conclusion}}
    };
  \end{tikzpicture}
\end{frame}

\begin{frame}{Conclusion}
  \begin{itemize}
    % \setlength\itemsep{0.8em} 
    % \setlength{\skip\footins}{2cm}
    \item Improvements to the original policy package provide a robust and efficient implementation of policy-based constraints
    \item The policy package is now capable of managing constraints across an entire class hierarchy
    \item The policy definitions are tightly paired with the class they constrain
    \item Macros reduce the expense of defining common policies while still allowing flexibility for custom policies
    \item A complete implementation is available in the Appendix of the paper and on GitHub\footnotemark\,\,which can be included directly in a project to start using policies immediately
  \end{itemize}
  \footnotetext[1]{\url{https://github.com/DillanCMills/policy_pkg}}
\end{frame}

% Questions
\begin{frame}
  \begin{tikzpicture}[overlay, remember picture]
    \node[] at (current page.center){
      \includegraphics[height=\paperheight]{images/image5.png}
    };
    \node[align=center, minimum width=0.7\textwidth, text width=1\textwidth] (title) at (current page.center) {
      {\color{white}\Huge{Questions?}}
    };
  \end{tikzpicture}
\end{frame}

\end{document}



%%%%%%%%%%%%%%%%%%%%%%%%%%%%%%%%%%%%%%%%%%%%%%%%%%
% Templates
%%%%%%%%%%%%%%%%%%%%%%%%%%%%%%%%%%%%%%%%%%%%%%%%%%

% Section slide
\begin{frame}
  \begin{tikzpicture}[overlay, remember picture]
    \node[] at (current page.center){
      \includegraphics[height=\paperheight]{images/image5.png}
    };
    \node[align=center, minimum width=0.7\textwidth, text width=1\textwidth] (title) at (current page.center) {
      {\color{white}\Huge{Slide Title} \\[10pt]
      \huge{Slide Subtitle}}
    };
  \end{tikzpicture}
\end{frame}

% Content slide
\begin{frame}{Title}
  \begin{itemize}
    \item Content
  \end{itemize}
\end{frame}

\begin{itemize}
  \item 
\end{itemize}