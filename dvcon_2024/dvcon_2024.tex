\documentclass[conference,onecolumn]{IEEEtran}
\usepackage{cite}
\usepackage{amsmath,amssymb,amsfonts}
\usepackage{algorithmic}
\usepackage{graphicx}
\usepackage{textcomp}
\usepackage{xcolor}
\usepackage{minted}
\renewcommand{\MintedPygmentize}{./pygments-plugin-systemverilog/venv/bin/pygmentize}
\usepackage[framemethod=TikZ]{mdframed}
\usepackage[T1]{fontenc}
\usepackage{inconsolata}
\usepackage{caption}
% \usepackage{hyperref}
\usepackage{ifthen}
% \usepackage{fancyhdr}

\def\BibTeX{{\rm B\kern-.05em{\sc i\kern-.025em b}\kern-.08em
T\kern-.1667em\lower.7ex\hbox{E}\kern-.125emX}}

\mdfsetup{
    middlelinewidth   = 2,
    roundcorner       = 10pt,
    leftmargin        = 40,
    rightmargin       = 40,
    backgroundcolor   = gray!05,
    middlelinecolor   = gray!30!white,
    innertopmargin    = \topskip,
    innerbottommargin = \topskip,
    splittopskip      = \topskip,
}

\usemintedstyle{sv-style-light}
\captionsetup{hypcap=false, position=top}

\newcommand{\svfile}[2]{{
\begin{mdframed}
\inputminted[
    linenos,
    numbersep=16pt,
    breaklines,
]{sv-lang}{#1}
\end{mdframed}
\ifthenelse{\equal{#2}{}}{}{\captionof{figure}{#2}}
}}

\newcommand{\svfilerange}[4]{{
\begin{mdframed}
\inputminted[
    linenos,
    numbersep=16pt,
    breaklines,
    firstline=#2,
    lastline=#3,
    firstnumber=1,
]{sv-lang}{#1}
\end{mdframed}
\ifthenelse{\equal{#4}{}}{}{\captionof{figure}{#4}}
}}

\newcommand{\code}[1]{
\textbf{\mintinline{text}{#1}}
}

\newminted[svcode]{sv-lang}{
    linenos,
    numbersep=16pt,
    breaklines,
}

\surroundwithmdframed{svcode}


\begin{document}

\title{Four Problems with Policy-Based Constraints and How to Fix Them}

\author{
    \IEEEauthorblockN{Dillan Mills}
    \IEEEauthorblockA{\textit{Synopsys, Inc.} \\
        Maricopa, AZ \\
        dillan@synopsys.com}
    \and
    \IEEEauthorblockN{Chip Haldane}
    \IEEEauthorblockA{\textit{The Chip Abides, LLC} \\
        Gilbert, AZ \\
        chip@thechipabides.com}
}

\maketitle

\thispagestyle{plain}
\pagestyle{plain}

\begin{abstract}
    This paper presents solutions to problems encountered in the implementation of policy classes for SystemVerilog constraint layering. Policy classes provide portable and reusable constraints that can be mixed and matched into the object being randomized. There have been many papers and presentations on policy classes since the original presentation by John Dickol at DVCon 2015. The paper addresses three problems shared by all public policy class implementations and presents a solution to a fourth problem. The proposed solutions introduce policy class inheritance, tightly pair policy definitions with the class they constrain, reduce the expense of defining common policies using macros, and demonstrate how to treat policies as disposable and lightweight objects. The paper concludes that the proposed solution improves the usability and efficiency of policy classes for SystemVerilog constraint layering.
\end{abstract}

\section{Constraints and Policy Class Review}

Random objects and constraints are the foundational building blocks of constrained random verification in SystemVerilog. The simplest implementations embed fixed constraints within a class definition.  Embedded constraints lack flexibility; all randomized object instances must meet these requirements exactly as they are written.

In-line constraints using the \code{with} construct offer marginally better flexibility.  Although these external constraints allow greater variability of random objects, their definitions are still fixed within the calling context.  Furthermore, all in-line constraints must be specified within a single call to \code{randomize()}.

Policy classes are a technique for applying SystemVerilog constraints in a portable, reusable, and incremental manner, originally described by John Dickol \cite{b1}\cite{b2}. The operating mechanism leverages an aspect of "global constraints," the simultaneous solving of constraints across a set of random objects.  Randomizing an object that contains policies also randomizes the policies.  Meanwhile, the policies contain a reference back to the container.  Consequently, the policy container is constrained by the policies it contains.  Dickol's approach is illustrated by the following code.

\svfile{example_code/v1_original_policies/policy_base.svh}{The \code{policy_base} and \code{policy_list} classes}

These two base classes provide the core definitions for policies: \code{policy_base} implements the hook back to the policy container, and \code{policy_list} enables related policies to be organized into groups. Both classes are parameterized by a container object type, so a unique specialization will be required for each policy-enabled container.  Policy containers like transactions, sequences, and configuration objects implement these classes to support flexible random steering.  Below are examples of a generic transaction, \code{addr_txn}, with a random address and size and some policies to constrain those attributes.

\svfile{example_code/v1_original_policies/addr_txn.svh}{The \code{addr_txn} class}

\svfile{example_code/v1_original_policies/addr_policies.svh}{Policies for constraining the \code{addr_txn} class}

The next three classes implement some policies for the \code{addr_txn} class. A list of valid address ranges can be stored in the \code{ranges} array. The \code{addr_permit_policy} will choose one of the ranges at random and constrain the address to be within the range, while the \code{addr_prohibit_policy} will exclude the address from randomizing inside any of the ranges in its list.


The final class shows how policies might be used. The \code{addr_constrained_txn} class extends \code{addr_txn} and defines two policies, one that permits the address to be within one of two ranges, and one that prohibits the address from being within a third range. The \code{addr_constrained_txn} class then passes the local \code{pcy} list to the parent \code{policy} queue.

\svfile{example_code/v1_original_policies/addr_constrained_txn.svh}{The \code{addr_constrained_txn} subclass}

At this point, an instance of \code{addr_constrained_txn} can be created and randomized like normal, and the address will be constrained based on the embedded policies.

{\begin{svcode}
addr_constrained_txn txn = new();
txn.randomize();
\end{svcode}
\captionof{figure}{Randomizing an instance of \code{addr_constrained_txn}}}

Further work on policy-based constraints has been presented since the original DVCon presentation in 2015. Kevin Vasconcellos and Jeff McNeal applied the concept to test configuration and added many nice utilities to the base policy class \cite{b3}. Chuck McClish extended the concept to manage real number values for User Defined Nettypes (UDN) and Unified Power Format (UPF) pins in an analog model \cite{b4}. Additionally, McClish defined a policy builder class that was used to generically build multiple types of policies while reducing repeated code that was shared between each policy class in the original implementation.

Although there has been extensive research on policies, this paper aims to address and provide solutions for three issues that have not been adequately resolved in previous implementations. Furthermore, a fourth problem that arose during testing of the upgraded policy package implementation will also be discussed and resolved.

\section{Problem \#1: Parameterized Policies}

The first problem with the above policy implementation is that because \code{policy_base} is parameterized to the class it constrains, different specializations cannot be grouped and indexed. The awkward consequences of this limitation become apparent when using policies with a class hierarchy. If you extend a class and add a new random field then you need a new policy type to constrain that field. The new policy type requires its own policy list, and the new list must be traversed and mapped back to the container during \code{pre_randomize()}.

Imagine a complex class hierarchy with several layers of inheritance and extension, and constrainable attributes on each layer (one common example is a multi-layered sequence API library, such as the example presented by Jeff Vance \cite{b5}). Using policies becomes cumbersome in this case because each class layer requires a unique family of constraints organized into a distinct list.  This stratification of polices places a burden on users to know which layer of the class hierarchy defines each attribute they want to constrain, and the name of the associated policy list for the matching policy type.  For example, extending the \code{addr_txn} class to create a version with parity checking results in a class hierarchy that looks like this:

\svfile{example_code/v2_extended_txns/addr_txn.svh}{Modified address transaction classes with a parity checking subclass and policies included}

Scaling this implementation results in a lot of repeated boilerplate code and is not very intuitive to use. Each additional subclass in a hierarchy only increases the chaos and complexity of implementing and using policies effectively.

The solution to this problem is to replace the parameterized policy base with a non-parameterized base and a parameterized extension.  We chose an interface class as our non-parameterized base for the flexibility it offers over a virtual base class---specifically, our policies are bound only to implement the interface functions and not to extend a specific class implementation.

\svfile{example_code/v3_extended_policies/policy_base.svh}{The \code{policy} interface class and \code{policy_imp} class}

A non-parameterized base enables all policies targeting a particular class hierarchy to be stored within a single common \code{policy_queue}.  A parameterized template, \code{policy_imp}, implements the base interface and core functionality required by all policies.

One consequence of eliminating the parameter from our base type is that the policy-enabled container object, \code{item}, and its assignment function, \code{set_item()}, are no longer strongly typed.  Here we make a small concession, using \code{uvm_object} as our default policy-enabled type.  This means that all classes that implement our policies must derive from \code{uvm_object}, and we need to use dynamic casting to ensure that policies and their containers are type-compatible.  In the example above, \code{item} is set to \code{null} and randomization is disabled when the cast fails, preventing runtime problems in the event that incompatible policies are applied.

Not much changes when it comes to defining policies; the address policies now extend \code{policy_imp} instead of \linebreak\code{policy_base}, and the underlying constraints are written as implications so that they will not apply when \code{item} is missing.

\svfile{example_code/v3_extended_policies/addr_policies.svh}{Address policies updated to use the new policy implementation}

However, the address transaction classes are simplified considerably. Only a single policy queue is required in the class hierarchy, and the vast majority of the boilerplate code has been eliminated, including all of the specialized lists. The \code{addr_txn} class now extends \code{uvm_object} to provide compatibility with the policy interface.

\svfile{example_code/v3_extended_policies/addr_txn.svh}{Address transaction classes updated to use the new policy implementation}


\section{Problem \#2: Definition Location}

The second problem with policies is ``where do I define my policy classes?'' This is not a complicated problem to solve; most users will likely wish to place their policy classes in a file or files close to the class they are constraining. However, directly embedding policy definitions within the class they constrain offers a myriad of benefits.  Not only does this convention eliminate all guesswork about where to define and discover policies, but embedded policies also gain access to all members of their container class, including protected properties and methods!  This privileged access enables policies to constrain attributes of a class that are not otherwise exposed, improving encapsulation.

To further optimize the organization of potentially large families of policies, we establish a convention of defining all policy classes within an embedded wrapper class called \code{POLICIES}.  Each layer of a class hierarchy that implements policies will have its own embedded \code{POLICIES} wrapper, and individual \code{POLICIES} wrappers extend other wrappers in a manner parallel to their container classes.  This parallel inheritance pattern is shown below, with \code{addr_p_txn::POLICIES} extending \code{addr_txn::POLICIES}.

\svfile{example_code/v4_embedded_policies/addr_txn.svh}{Address transaction classes with embedded policies}

This example also shows how we can define static constructor functions within \code{POLICIES} wrappers. This practice further reduces the cost of using policies since we can instantiate and initialize them with a single call, as demonstrated with the call to \linebreak\code{addr_constrained_txn::POLICIES::PARITY_ERR()}. Note that although the \code{PARITY_ERR} constructor is defined in \linebreak\code{addr_p_txn::POLICIES}, it is accessible through \code{addr_constrained_txn::POLICIES} because of the wrapper class inheritance. The \code{POLICIES::} scoping layer even helps to make code more readable and easy to understand.

What's more, the \code{parity_err} property has now been defined as \code{protected}, preventing anything but our \code{PARITY_ERR} policy from manipulating that "knob."  In fact, a more advanced use of policies might define all members of a target class as protected, restricting the setting of fields exclusively through policies and reading through accessor functions, thus encouraging maximum encapsulation/loose coupling, which reduces the cost of maintaining and enhancing code and prevents bugs from cascading into classes that use policy-enabled classes.

\section{Problem \#3: Boilerplate Overload}

The third problem with using policies is that policies are relatively expensive to define since you need at a minimum: a class definition, a constructor, and a constraint. This will be relatively unavoidable for complex policies, such as those defining a relationship between multiple specific class attributes. For generic policies, such as equality constraints (property equals X), range constraints (property between Y and Z), or set membership (keyword \code{inside}) constraints, macros can be used to drastically reduces the expense and risk of defining common policies. The macros are responsible for creating the specialized policy class for the required constraint within the target class, as well as a static constructor function that is used to create new policy instances of the class. Additional macros are utilized for setting up the embedded \code{POLICIES} class within the target class. These macros can be used hand in hand with the non-macro policy classes needed for complex constraints, if necessary.

\svfile{example_code/v5_policy_macros/policy_macros.svh}{Macros for setting up the embedded \code{POLICIES} class and a fixed value policy}

This example includes a \code{`fixed_policy} macro, which wraps two additional macros responsible for creating a policy class and a static constructor for the class. This \code{`fixed_policy} example policy class lets you constrain a property to a fixed value. A more complete macro definition can be found in Appendix \ref{policy_macros}. The appendix includes \code{`start_policies}, \code{`start_extended_policies}, and \code{`end_policies} macros that are used to create the embedded \code{POLICIES} class within the constrained class instead of using hard-coded \code{class} and \code{endclass} statements. They set up class inheritance as needed and create a local typedef for the \code{policy_imp} parameterized type. 

\svfile{example_code/v5_policy_macros/addr_txn.svh}{Simplified address transaction classes using the policy macros}

The base \code{addr_txn} class has complex policies with a relationship between the \code{addr} and \code{size} fields, so rather than creating a policy macro that will only be used once, they can either be left as-is within the embedded policies class, or moved to a separate file and included with \code{`include} as was done here to keep the transaction class simple. The child parity transaction class is able to use the \code{`fixed_policy} macro to constrain the \code{parity_err} field. The constraint block remains the same as the previous example.

\section{Problem \#4: Unexpected Policy Reuse Behavior and Optimizing for Lightweight Policies}

A fourth problem occurred during our initial deployment of policies.  We observed occasional unexpected behavior when attempting to re-randomize objects with policies.  We didn't thoroughly characterize the behavior, but sometimes policies seemed to ``remember'' previous randomizations and wouldn't reapply their constraints during subsequent \code{randomize} calls.  Results would clearly violate even simple policies.

Our policy architecture prioritizes scalability; policy classes are lightweight with a minimal footprint.  Rather than investing effort to diagnose and work around the problem with reusing policies, we adopted a ``use once and discard'' approach, leaning into their disposable nature.  It costs little to apply fresh policy instances before re-randomizing a target object.  Following this strategy completely eliminated policy misbehavior.

To facilitate a safer form of policy reuse we introduced a \code{copy} method that returns a fresh policy instance initialized to the same state as the policy that implements it.  We also doubled down on our use of static constructors to generate initialized policies.

{\begin{svcode}
static function addr_parity_err_policy PARITY_ERR(bit value);
    PARITY_ERR = new(value);
endfunction
...
this.policies = '{addr_constrained_txn::POLICIES::PARITY_ERR(1'b1)};
\end{svcode}
\captionof{figure}{Example static constructor function from the \code{PARITY_ERR} policy class}}

Passing array literals populated by policy instances from static constructors proved to be an excellent way to pack a lot of intent into little code.  It also neatly worked around the reliability issues of reused policies.

\section{More Improvements to the Policy Package}

The examples presented so far are functional, but are lacking many features that would be useful in a real-world implementation. The following examples will present additional improvements to the policy package that will make it more practical and efficient to use.

\subsection{Expanding the \code{policy} interface class}

The following \code{policy} interface class adds additional methods for managing a policy.

\svfile{../policy_pkg/policy.svh}{Expanded \code{policy} interface class}

The \code{name}, \code{description}, and \code{copy} methods are implemented by the policy (or policy macro) directly and provide reporting information useful when printing messages about the policy to the log for the former two, or specific behavior for making a copy for the latter. The remaining three methods are implemented by \code{policy_imp} and are shared by all policies. 

\subsection{Better type safety checking and reporting in \code{policy_imp} methods}

Some of the benefits of above methods can be seen by examining the new \code{set_item} method used by \code{policy_imp}.

{\begin{svcode}
virtual function void set_item(uvm_object item);
    if (item == null) begin
        `uvm_error("policy::set_item()", "NULL item passed")

    end else if ((this.item_is_compatible(item)) && $cast(this.m_item, item)) begin
        `uvm_info(
            "policy::set_item()",
            $sformatf(
                "policy <%s> applied to item <%s>: %s",
                this.name(), item.get_name(), this.description()
            ),
            UVM_FULL
        )
        this.m_item.rand_mode( 1 );

    end else begin
        `uvm_warning(
            "policy::set_item()",
            $sformatf(
                "Item <%s> type <%s> is not compatible with policy <%s> type <%s>",
                item.get_name(), item.get_type_name(), this.name(), this.type_name()
            )
        )
        this.m_item = null;
        this.m_item.rand_mode( 0 );
    end
endfunction: set_item
\end{svcode}
\captionof{figure}{\code{set_item} method from \code{policy_imp}}}

The \code{set_item} method makes use of all the reporting methods to provide detailed log messages when \code{set_item} succeeds or fails. Additionally, the \code{item_is_compatible} is used before the \code{\$cast} method is called and the \code{rand_mode} state is kept consistent with the result of the cast.

\subsection{Replacing \code{policy_list} with \code{policy_queue}}

Eagle-eyed readers might have noticed the lack of presence of a \code{policy_list} class in any of the examples above after migrating to the improved policy interface. Rather, a single typedef is all that is necessary to manage policies in a class.

{\begin{svcode}
typedef policy policy_queue[$];
\end{svcode}
\captionof{figure}{The \code{policy_queue} typedef}}

The \code{policy_queue} type is capable of storing any policy that implements the \code{policy} interface. The default queue methods are sufficient for aggregating policies, and in practice we found that using policy queues as containers was more efficient than \code{policy_list} instances. For example, for functions expecting a \code{policy_queue} argument we can directly pass in array literals populated by calls to static constructor functions, allowing us to define, initialize, aggregate, and pass policies all in a single line of code!

\subsection{Provide a common implementation for policy-enabled classes using the \code{policy_container} interface and \code{policy_object} mixin}

Our solution encourages loosely coupled code by defining APIs that pass \code{policy_queue} arguments and hide a target object's policies queue from other classes.

\svfile{../policy_pkg/policy_container.svh}{The \code{policy_container} interface class}

This interface class should be implemented by any class that needs to use policies. The method implementations can then be used to manipulate a policy queue as needed, including adding to, replacing, or clearing the queue, and returning a handle to the queue or a copy of the queue to use elsewhere.

The \code{policy_container} interface class allows for a base class implementation that can be used to manage constraints across the entire class hierarchy. 

\svfile{example_code/v5_policy_macros/policy_object.svh}{A \code{policy_object} base class implementation}

This base class implements all the \code{policy_container} functionality (a complete example is available in Appendix \ref{policy_pkg}). Packaging a common implementation and interface into a mixin enables sharing a common implementation and API for policy support among any classes that might benefit from the use of policies, as seen in the following example.

{\begin{svcode}
// Use policy_object for transactions
class base_txn extends policy_object #(uvm_sequence_item);

// Use policy_object for sequences
class base_seq #(type REQ=uvm_sequence_item, RSP=REQ) extends policy_object #( uvm_sequence#(REQ, RSP) );

// Use policy_object for configuration objects
class cfg_object extends policy_object #(uvm_object);
\end{svcode}
\captionof{figure}{Example classes using the \code{policy_object} base class}}

\subsection{Protecting the policy queue enforces loosely coupled code}

A subtle but significant additional benefit to using a base \code{policy_object} class along with the \code{policy_container} API is the ability to mark the container's policy queue as protected and prevent direct access to it. 

The original implementation called \code{set_item} on each policy during the \code{pre_randomize} stage. That was necessary because the \code{policy_queue} was public, so there was nothing to prevent callers from adding policies without linking them to the target class.

Using an interface class and making the implementation private means the callers may only set or add policies using the available interface class routines, and because those routines are solely responsible for applying policies, they can also check compatibility (and filter incompatible policies) and call \code{set_item} immediately. This can be seen in the example \code{policy_object} implementation above, in the usage of the protected function \code{try_add_policy}.

By calling \code{set_item} when the policy is added to the queue, all policies will be associated with the target item automatically, so there is no need to do it during \code{pre_randomize}. This eliminates an easily-overlooked requirement for classes extending \code{policy_object} to make sure they call \code{super.pre_randomize()} in their local \code{pre_randomize} function.

\section{Conclusion}

The improvements to the policy package presented in this paper provide a more robust and efficient implementation of policy-based constraints for SystemVerilog. The policy package is now capable of managing constraints across an entire class hierarchy, and the policy definitions are tightly paired with the class they constrain. The use of macros reduces the expense of defining common policies, while still allowing great flexibility in any custom policies necessary.

The complete policy package is available in Appendix \ref{policy_pkg}, with additional macro examples available in Appendix \ref{policy_macros}. A functional package is also available for download \cite{b6} which can be included directly in a project to start using policies immediately.

\begin{thebibliography}{00}
    \bibitem{b1} J. Dickol, ``SystemVerilog Constraint Layering via Reusable Randomization Policy Classes,'' DVCon, 2015.
    \bibitem{b2} J. Dickol, ``Complex Constraints: Unleashing the Power of the VCS Constraint Solver,'' SNUG Austin, September 29, 2016.
    \bibitem{b3} K. Vasconcellos, J. McNeal, ``Configuration Conundrum: Managing Test Configuration with a Bite-Sized Solution,'' DVCon, 2021.
    \bibitem{b4} C. McClish, ``Bi-Directional UVM Agents and Complex Stimulus Generation for UDN and UPF Pins,'' DVCon, 2021.
    \bibitem{b5} J. Vance, J. Montesano, M. Litterick, J. Sprott, ``Be a Sequence Pro to Avoid Bad Con Sequences,'' DVCon, 2019.
    \bibitem{b6} D. Mills, C. Haldane, ``policy\_pkg Source Code,'' https://github.com/DillanCMills/policy\_pkg.
\end{thebibliography}

\newpage
\appendices

\section{Policy Package}\label{policy_pkg}

\subsection{policy\_pkg.sv}\label{policy_pkg.sv}
\svfile{../policy_pkg/policy_pkg.sv}{}

\subsection{policy.svh}\label{policy.svh}
\svfile{../policy_pkg/policy.svh}{}

\subsection{policy\_imp.svh}\label{policy_imp.svh}
\svfile{../policy_pkg/policy_imp.svh}{}

\subsection{policy\_container.svh}\label{policy_container.svh}
\svfile{../policy_pkg/policy_container.svh}{}

\subsection{policy\_object.svh}\label{policy_object.svh}
\svfile{../policy_pkg/policy_object.svh}{}

\newpage
\section{Policy Macros}\label{policy_macros}

\subsection{policy\_macros.svh}\label{policy_macros.svh}
\svfile{../policy_pkg/macros/policy_macros.svh}{}

\subsection{constant\_policy.svh}\label{constant_policy.svh}
\svfile{../policy_pkg/macros/constant_policy.svh}{}

\subsection{fixed\_policy.svh}\label{fixed_policy.svh}
\svfile{../policy_pkg/macros/fixed_policy.svh}{}

\subsection{member\_policy.svh}\label{member_policy.svh}
\svfile{../policy_pkg/macros/member_policy.svh}{}

\subsection{range\_policy.svh}\label{range_policy.svh}
\svfile{../policy_pkg/macros/range_policy.svh}{}

\end{document}
