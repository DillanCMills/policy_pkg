\documentclass[conference,onecolumn]{IEEEtran}
% \IEEEoverridecommandlockouts
% The preceding line is only needed to identify funding in the first footnote. If that is unneeded, please comment it out.
\usepackage{cite}
\usepackage{amsmath,amssymb,amsfonts}
\usepackage{algorithmic}
\usepackage{graphicx}
\usepackage{textcomp}
\usepackage{xcolor}
\usepackage{minted}
\usepackage[framemethod=TikZ]{mdframed}
\usepackage[T1]{fontenc}
\usepackage{inconsolata}

\def\BibTeX{{\rm B\kern-.05em{\sc i\kern-.025em b}\kern-.08em
T\kern-.1667em\lower.7ex\hbox{E}\kern-.125emX}}

\mdfsetup{
    middlelinewidth    = 2,
    roundcorner       = 10pt,
    leftmargin        = 40,
    rightmargin       = 40,
    backgroundcolor   = black!95,
    middlelinecolor    = gray!50,
    innertopmargin    = \topskip,
    innerbottommargin = \topskip,
    splittopskip      = \topskip,
}

\usemintedstyle{one-dark}

\newcommand{\svfile}[1]{
\begin{mdframed}
\inputminted[
    linenos,
    numbersep=16pt,
    breaklines,
]{CustomSystemVerilogLexer.py:CustomSystemVerilogLexer -x}{#1}
\end{mdframed}
}
\newcommand{\code}[1]{
\textbf{\mintinline{text}{#1}}
}

\newminted[svcode]{CustomSystemVerilogLexer.py:CustomSystemVerilogLexer -x}{
    linenos,
    numbersep=16pt,
    breaklines,
}
\surroundwithmdframed{svcode}

\begin{document}

\title{Four Problems with Policy-Based Constraints and How to Fix Them}

\author{
    \IEEEauthorblockN{Dillan Mills}
    \IEEEauthorblockA{\textit{Synopsys, Inc.} \\
        Maricopa, AZ \\
        dillan@synopsys.com}
    \and
    \IEEEauthorblockN{Chip Haldane}
    \IEEEauthorblockA{\textit{The Chip Abides, LLC} \\
        Gilbert, AZ \\
        chip@thechipabides.com}
}

\maketitle

\begin{abstract}
    This paper presents solutions to problems encountered in the implementation of policy classes for SystemVerilog constraint layering. Policy classes provide portable and reusable constraints that can be mixed and matched into the object being randomized. There have been many papers and presentations on policy classes since the original presentation by John Dickol at DVCon 2015. The paper addresses three problems shared by all public policy class implementations and presents a solution to a fourth problem. The proposed solutions introduce policy class inheritance, tightly pair policy definitions with the class they constrain, reduce the expense of defining common policies using macros, and demonstrate how to treat policies as disposable and lightweight objects. The paper concludes that the proposed solution improves the usability and efficiency of policy classes for SystemVerilog constraint layering.
\end{abstract}

\section{Constraints and Policy Class Review}

When working with randomized objects in SystemVerilog, it is typically desired to control the randomization of the item being randomized. The common method of doing so is by using constrained random value generation. Constraints for the randomization of class properties can be provided within the class definition, which will require randomized properties to always pass the specified constaint check. For example, the following \code{addr_txn} class contains a constraint on the \code{addr} property to ensure it is always word aligned.

\begin{svcode}
class addr_txn;
    rand bit [31:0] addr;
    rand int        size;

    constraint c_size {size inside {1, 2, 4};}
endclass
\end{svcode}

Constraints can also be provided outside the class definition at the point of randomization by using the \code{with} construct. This allows variation to the constraints based on the context of the randomization. For example, the same \code{addr_txn} class can have an instance of it randomized with the following constraint on its \code{data} property.

\begin{svcode}
addr_txn txn = new();
txn.randomize() with {addr == 32'hdeadbeef;};
\end{svcode}

Policy classes are a technique for SystemVerilog constraint layering with comparable performance to traditional constraint methods, while additionally providing portable and reusable constraints that can be mixed and matched into the object being randomized, originally presented by John Dickol \cite{b1}\cite{b2}. This is possible through SystemVerilog ``Global Constraints'', which randomizes all lower-level objects whenever a top-level object is randomized, and solves all rand variables and constraints across the entire object hierarchy simultaneously. A summary of his approach can be found by examining the following code.

\svfile{example_code/v1_original_policies/policy_base.svh}

These first two classes provide the base class types for the actual policies. This allows different, related policies to be grouped together in \code{policy_list} and applied to the randomization call concurrently. Both of these classes are parameterized to a specific object type, so separate instances will be required for each class type that needs to be constrained.


\svfile{example_code/v1_original_policies/addr_policies.svh}

The next three classes contain implementations of the \code{policy_base} class for the \code{addr_txn} class. A list of valid address ranges can be stored in the \code{ranges} array. The \code{addr_permit_policy} will choose one of the ranges at random and constrain the address to be within the range, while the \code{addr_prohibit_policy} will exclude the address from randomizing inside any of the ranges in its list.


\svfile{example_code/v1_original_policies/addr_txn.svh}

The final two classes show how these policies are used. The \code{addr_txn} class creates a \code{policy} queue, and during the \code{pre_randomize} function sets the handle to the \code{item} object within each policy to itself. The \code{addr_constrained_txn} subclass adds two policies to a local \code{pcy} list, one that permits the address to be within one of two ranges, and one that prohibits the address from being within a third range. The \code{addr_constrained_txn} class can then passes the local \code{pcy} list to the parent \code{policy} queue.

At this point, an instance of \code{addr_constrained_txn} can be created and randomized like normal, and the address will be constrained based on the policies provided.

\begin{svcode}
addr_constrained_txn txn = new();
txn.randomize();
\end{svcode}

Further work on policy-based constraints has been presented since the original DVCon presentation in 2015. Kevin Vasconcellos and Jeff McNeal applied the concept to test configuration and added many nice utilities to the base policy class \cite{b3}. Chuck McClish extended the concept to manage real number values for User Defined Nettypes (UDN) and Unified Power Format (UPF) pins in an analog model \cite{b4}. Additionally, he defined a policy builder class that was used to generically build multiple types of policies while reducing repeated code that was shared between each policy class in the original implementation.

Although there has been extensive research on policies, this paper aims to address and provide solutions for three issues that have not been adequately resolved in previous implementations. Furthermore, a fourth problem that arose during testing of the upgraded policy package implementation will also be discussed and resolved.

\section{Problem \#1: Parameterized Policies}

The first problem with the above policy implementation is that because policies are specialized using parameterization to the class they constrain, different specializations of parameterized classes cannot be grouped and indexed. This is most apparent when creating a class hierarchy. If you extend \code{addr_txn} and add additional fields to be constrained, then you need a new policy type. The new policy type requires an extra policy list to modify any new attributes in the child class and requires the user to know which layer of the class hierarchy defines each attribute they want to constrain. Imagine a complex class hierarchy with many layers of inheritance and extension, and new constrainable attributes on each layer (one common example is a multi-layered sequence API library, such as the example presented by Jeff Vance \cite{b5}). Using policies becomes cumbersome in this case because you need to manage a separate policy list for each level of the hierarchy and know what each list can and cannot contain. You could end up with a final class that looks like this:

\begin{svcode}
class addr_l2_txn extends addr_l1_txn;
  rand int f2;
  rand policy_base#(addr_l2_txn) policy_l2[$];

  function void pre_randomize;
    super.pre_randomize();
    foreach(policy_l2[i]) policy_l2[i].set_item(this);
  endfunction
endclass

// ... repeat for addr_l3_txn and addr_l4_txn

class addr_constrained_txn extends addr_l4_txn;
    function new;
        addr_permit_policy        permit = new;
        addr_prohibit_policy      prohibit = new;
        addr_l2_policy            l2 = new;
        addr_l3_policy            l3 = new;
        addr_l4_policy            l4 = new;
        policy_list#(addr_l1_txn) pcy_l1 = new;
        policy_list#(addr_l2_txn) pcy_l2 = new;
        policy_list#(addr_l3_txn) pcy_l3 = new;
        policy_list#(addr_l4_txn) pcy_l4 = new;

        permit.add('h00000000, 'h0000FFFF); 
        permit.add('h10000000, 'h1FFFFFFF); 
        pcy_l1.add(permit);
        prohibit.add('h13000000, 'h130FFFFF); 
        pcy_l1.add(prohibit);

        l2.set('h12345678); pcy_l2.add(l2);
        l3.set('h55555555); pcy_l3.add(l3);
        l4.set('h87654321); pcy_l4.add(l4);

        this.policy_l1 = {pcy_l1};
        this.policy_l2 = {pcy_l2};
        this.policy_l3 = {pcy_l3};
        this.policy_l4 = {pcy_l4};
    endfunction
endclass
\end{svcode}

This class ends up with a lot of boilerplate code and is not very intuitive to use. The user needs to keep track of what the difference is between each policy list and apply constraints properly. Additionally, more inheritance causes the class to grow further. This would strongly discourage using inheritance, which is not desirable.

The solution to this problem is to introduce class inheritance into the policy classes that mirror the target class hierarchy. So for the four-layered \code{addr_txn} example, four matching policy classes would be created to match.

\begin{svcode}
class addr_l1_policy extends policy_base#(addr_l1_txn);
    addr_range ranges[$];

    function add(addr_t min, addr_t max);
        addr_range rng = new(min, max);
        ranges.push_back(rng);
    endfunction
endclass

class addr_l2_policy extends addr_l1_policy;
    // ... additional fields that are used to constrain the object
endclass

class addr_l3_policy extends addr_l2_policy;
    // ... additional fields that are used to constrain the object
endclass

class addr_l4_policy extends addr_l3_policy;
    // ... additional fields that are used to constrain the object
endclass
\end{svcode}


, supplemented by convenient macros to keep the solution simple.

\section{Problem \#2: Definition Location}

The second problem is ``where do I define my policy classes?'' This is not a complicated problem to solve, and most users likely place their policy classes in a file close to the class they are constraining. However, this paper will present a solution that explicitly pairs policy definitions with the class they constrain, so users never need to hunt for the definitions or guess where they are located, and so the definitions have an appropriate and consistent home in the environment. This is done by embedding a \code{POLICIES} class inside the class being constrained. This class acts as a container for all relevant policies and allows policies to be created by simply referencing the embedded class through the \code{::|}operator.

\section{Problem \#3: Boilerplate Overload}

The third problem is that policies are relatively expensive to define since you need at a minimum, a class definition, a constructor, and a constraint. This will be relatively unavoidable for complex policies, such as those defining a relationship between multiple specific class attributes. For generic policies, such as equality constraints, range constraints, or set membership (keyword \code{inside}) constraints, a strategy using macros is presented that drastically reduces the expense and risk of defining common policies. The macros are responsible for creating the specialized policy class for the required constraint within the target class, as well as a static constructor function that is used to create new policy instances of the class. Additional macros are utilized for setting up the embedded \code{POLICIES} class within the target class. These macros can be used hand in hand with the non-macro policy classes needed for the complex constraints, if necessary.

\section{Problem \#4: Unexpected Policy Reuse Behavior}

A fourth problem we noticed in our usage of policy classes was occasional unexpected behavior when attempting to reuse a policy. It appeared as though the policies ``remembered'' being randomized and wouldn't reapply their constraints during subsequent \code{randomize} calls. An exact diagnosis of the issue was not discovered because we transitioned our policy approach to a ``use once and discard'' approach, where we would treat a policy as disposable and model them as extremely lightweight classes. We implemented practices such as static functions to create new instances of the policies and policy methods to return fresh copies of themselves.

\section{Conclusion}

Functional example code will be provided to demonstrate the solutions to these problems and provide other minor improvements to the original policy class implementation.

\appendices

\section{Policy Package}

\subsection{policy\_pkg.sv}
\svfile{../policy_pkg/policy_pkg.sv}

\subsection{policy.svh}
\svfile{../policy_pkg/policy.svh}

\subsection{policy\_container.svh}
\svfile{../policy_pkg/policy_container.svh}

\subsection{policy\_imp.svh}
\svfile{../policy_pkg/policy_imp.svh}

\subsection{policy\_list.svh}
\svfile{../policy_pkg/policy_list.svh}

\section{Policy Macros}

\subsection{policy\_macros.svh}
\svfile{../policy_pkg/macros/policy_macros.svh}

\subsection{constant\_policy.svh}
\svfile{../policy_pkg/macros/constant_policy.svh}

\subsection{fixed\_policy.svh}
\svfile{../policy_pkg/macros/fixed_policy.svh}

\subsection{ranged\_policy.svh}
\svfile{../policy_pkg/macros/ranged_policy.svh}

\subsection{set\_policy.svh}
\svfile{../policy_pkg/macros/set_policy.svh}


% \section{Introduction}
% This document is a model and instructions for \LaTeX.
% Please observe the conference page limits.

% \section{Ease of Use}

% \subsection{Maintaining the Integrity of the Specifications}

% The IEEEtran class file is used to format your paper and style the text. All margins,
% column widths, line spaces, and text fonts are prescribed; please do not
% alter them. You may note peculiarities. For example, the head margin
% measures proportionately more than is customary. This measurement
% and others are deliberate, using specifications that anticipate your paper
% as one part of the entire proceedings, and not as an independent document.
% Please do not revise any of the current designations.

% \section{Prepare Your Paper Before Styling}
% Before you begin to format your paper, first write and save the content as a
% separate text file. Complete all content and organizational editing before
% formatting. Please note sections \ref{AA}--\ref{SCM} below for more information on
% proofreading, spelling and grammar.

% Keep your text and graphic files separate until after the text has been
% formatted and styled. Do not number text heads---{\LaTeX} will do that
% for you.

% \subsection{Abbreviations and Acronyms}\label{AA}
% Define abbreviations and acronyms the first time they are used in the text,
% even after they have been defined in the abstract. Abbreviations such as
% IEEE, SI, MKS, CGS, ac, dc, and rms do not have to be defined. Do not use
% abbreviations in the title or heads unless they are unavoidable.

% \subsection{Units}
% \begin{itemize}
%     \item Use either SI (MKS) or CGS as primary units. (SI units are encouraged.) English units may be used as secondary units (in parentheses). An exception would be the use of English units as identifiers in trade, such as ``3.5-inch disk drive''.
%     \item Avoid combining SI and CGS units, such as current in amperes and magnetic field in oersteds. This often leads to confusion because equations do not balance dimensionally. If you must use mixed units, clearly state the units for each quantity that you use in an equation.
%     \item Do not mix complete spellings and abbreviations of units: ``Wb/m\textsuperscript{2}'' or ``webers per square meter'', not ``webers/m\textsuperscript{2}''. Spell out units when they appear in text: ``. . . a few henries'', not ``. . . a few H''.
%     \item Use a zero before decimal points: ``0.25'', not ``.25''. Use ``cm\textsuperscript{3}'', not ``cc''.)
% \end{itemize}

% \subsection{Equations}
% Number equations consecutively. To make your
% equations more compact, you may use the solidus (~/~), the exp function, or
% appropriate exponents. Italicize Roman symbols for quantities and variables,
% but not Greek symbols. Use a long dash rather than a hyphen for a minus
% sign. Punctuate equations with commas or periods when they are part of a
% sentence, as in:
% \begin{equation}
%     a+b=\gamma\label{eq}
% \end{equation}

% Be sure that the
% symbols in your equation have been defined before or immediately following
% the equation. Use ``\eqref{eq}'', not ``Eq.~\eqref{eq}'' or ``equation \eqref{eq}'', except at
% the beginning of a sentence: ``Equation \eqref{eq} is . . .''

% \subsection{\LaTeX-Specific Advice}

% Please use ``soft'' (e.g., \verb|\eqref{Eq}|) cross references instead
% of ``hard'' references (e.g., \verb|(1)|). That will make it possible
% to combine sections, add equations, or change the order of figures or
% citations without having to go through the file line by line.

% Please don't use the \verb|{eqnarray}| equation environment. Use
% \verb|{align}| or \verb|{IEEEeqnarray}| instead. The \verb|{eqnarray}|
% environment leaves unsightly spaces around relation symbols.

% Please note that the \verb|{subequations}| environment in {\LaTeX}
% will increment the main equation counter even when there are no
% equation numbers displayed. If you forget that, you might write an
% article in which the equation numbers skip from (17) to (20), causing
% the copy editors to wonder if you've discovered a new method of
% counting.

%     {\BibTeX} does not work by magic. It doesn't get the bibliographic
% data from thin air but from .bib files. If you use {\BibTeX} to produce a
% bibliography you must send the .bib files.

%     {\LaTeX} can't read your mind. If you assign the same label to a
% subsubsection and a table, you might find that Table I has been cross
% referenced as Table IV-B3.

% {\LaTeX} does not have precognitive abilities. If you put a
% \verb|\label| command before the command that updates the counter it's
% supposed to be using, the label will pick up the last counter to be
% cross referenced instead. In particular, a \verb|\label| command
% should not go before the caption of a figure or a table.

% Do not use \verb|\nonumber| inside the \verb|{array}| environment. It
% will not stop equation numbers inside \verb|{array}| (there won't be
% any anyway) and it might stop a wanted equation number in the
% surrounding equation.

% \subsection{Some Common Mistakes}\label{SCM}
% \begin{itemize}
%     \item The word ``data'' is plural, not singular.
%     \item The subscript for the permeability of vacuum $\mu_{0}$, and other common scientific constants, is zero with subscript formatting, not a lowercase letter ``o''.
%     \item In American English, commas, semicolons, periods, question and exclamation marks are located within quotation marks only when a complete thought or name is cited, such as a title or full quotation. When quotation marks are used, instead of a bold or italic typeface, to highlight a word or phrase, punctuation should appear outside of the quotation marks. A parenthetical phrase or statement at the end of a sentence is punctuated outside of the closing parenthesis (like this). (A parenthetical sentence is punctuated within the parentheses.)
%     \item A graph within a graph is an ``inset'', not an ``insert''. The word alternatively is preferred to the word ``alternately'' (unless you really mean something that alternates).
%     \item Do not use the word ``essentially'' to mean ``approximately'' or ``effectively''.
%     \item In your paper title, if the words ``that uses'' can accurately replace the word ``using'', capitalize the ``u''; if not, keep using lower-cased.
%     \item Be aware of the different meanings of the homophones ``affect'' and ``effect'', ``complement'' and ``compliment'', ``discreet'' and ``discrete'', ``principal'' and ``principle''.
%     \item Do not confuse ``imply'' and ``infer''.
%     \item The prefix ``non'' is not a word; it should be joined to the word it modifies, usually without a hyphen.
%     \item There is no period after the ``et'' in the Latin abbreviation ``et al.''.
%     \item The abbreviation ``i.e.'' means ``that is'', and the abbreviation ``e.g.'' means ``for example''.
% \end{itemize}
% An excellent style manual for science writers is \cite{b7}.

% \subsection{Authors and Affiliations}
% \textbf{The class file is designed for, but not limited to, six authors.} A
% minimum of one author is required for all conference articles. Author names
% should be listed starting from left to right and then moving down to the
% next line. This is the author sequence that will be used in future citations
% and by indexing services. Names should not be listed in columns nor group by
% affiliation. Please keep your affiliations as succinct as possible (for
% example, do not differentiate among departments of the same organization).

% \subsection{Identify the Headings}
% Headings, or heads, are organizational devices that guide the reader through
% your paper. There are two types: component heads and text heads.

% Component heads identify the different components of your paper and are not
% topically subordinate to each other. Examples include Acknowledgments and
% References and, for these, the correct style to use is ``Heading 5''. Use
% ``figure caption'' for your Figure captions, and ``table head'' for your
% table title. Run-in heads, such as ``Abstract'', will require you to apply a
% style (in this case, italic) in addition to the style provided by the drop
% down menu to differentiate the head from the text.

% Text heads organize the topics on a relational, hierarchical basis. For
% example, the paper title is the primary text head because all subsequent
% material relates and elaborates on this one topic. If there are two or more
% sub-topics, the next level head (uppercase Roman numerals) should be used
% and, conversely, if there are not at least two sub-topics, then no subheads
% should be introduced.

% \subsection{Figures and Tables}
% \paragraph{Positioning Figures and Tables} Place figures and tables at the top and
% bottom of columns. Avoid placing them in the middle of columns. Large
% figures and tables may span across both columns. Figure captions should be
% below the figures; table heads should appear above the tables. Insert
% figures and tables after they are cited in the text. Use the abbreviation
% ``Fig.~\ref{fig}'', even at the beginning of a sentence.

% \begin{table}[htbp]
%     \caption{Table Type Styles}
%     \begin{center}
%         \begin{tabular}{|c|c|c|c|}
%             \hline
%             \textbf{Table} & \multicolumn{3}{|c|}{\textbf{Table Column Head}}                                                         \\
%             \cline{2-4}
%             \textbf{Head}  & \textbf{\textit{Table column subhead}}           & \textbf{\textit{Subhead}} & \textbf{\textit{Subhead}} \\
%             \hline
%             copy           & More table copy$^{\mathrm{a}}$                   &                           &                           \\
%             \hline
%             \multicolumn{4}{l}{$^{\mathrm{a}}$Sample of a Table footnote.}
%         \end{tabular}
%         \label{tab1}
%     \end{center}
% \end{table}

% \begin{figure}[htbp]
%     \centerline{\includegraphics{fig1.png}}
%     \caption{Example of a figure caption.}
%     \label{fig}
% \end{figure}

% Figure Labels: Use 8 point Times New Roman for Figure labels. Use words
% rather than symbols or abbreviations when writing Figure axis labels to
% avoid confusing the reader. As an example, write the quantity
% ``Magnetization'', or ``Magnetization, M'', not just ``M''. If including
% units in the label, present them within parentheses. Do not label axes only
% with units. In the example, write ``Magnetization (A/m)'' or ``Magnetization
% \{A[m(1)]\}'', not just ``A/m''. Do not label axes with a ratio of
% quantities and units. For example, write ``Temperature (K)'', not
% ``Temperature/K''.

% \section*{Acknowledgment}

% The preferred spelling of the word ``acknowledgment'' in America is without
% an ``e'' after the ``g''. Avoid the stilted expression ``one of us (R. B.
% G.) thanks $\ldots$''. Instead, try ``R. B. G. thanks$\ldots$''. Put sponsor
% acknowledgments in the unnumbered footnote on the first page.

\begin{thebibliography}{00}
    \bibitem{b1} J. Dickol, ``SystemVerilog Constraint Layering via Reusable Randomization Policy Classes,'' DVCon, 2015.
    \bibitem{b2} J. Dickol, ``Complex Constraints: Unleashing the Power of the VCS Constraint Solver,'' SNUG Austin, September 29, 2016.
    \bibitem{b3} K. Vasconcellos, J. McNeal, ``Configuration Conundrum: Managing Test Configuration with a Bite-Sized Solution,'' DVCon, 2021.
    \bibitem{b4} C. McClish, ``Bi-Directional UVM Agents and Complex Stimulus Generation for UDN and UPF Pins,'' DVCon, 2021.
    \bibitem{b5} J. Vance, J. Montesano, M. Litterick, J. Sprott, ``Be a Sequence Pro to Avoid Bad Con Sequences,'' DVCon, 2019.
\end{thebibliography}
\end{document}
